\documentclass[10pt,landscape,a4paper]{article}
\usepackage{multicol}
\usepackage{calc}
\usepackage{ifthen}
\usepackage[utf8]{inputenc}
\usepackage[T1]{fontenc}
\usepackage[french]{babel}
\usepackage[landscape]{geometry}
\usepackage{nicefrac}
\usepackage{ulem}
\usepackage{amsmath,amsthm,amsfonts,amssymb}
\usepackage{color,graphicx,overpic}
\usepackage{hyperref}


% This sets page margins to .5 inch if using letter paper, and to 1cm
% if using A4 paper. (This probably isn't strictly necessary.)
% If using another size paper, use default 1cm margins.
\ifthenelse{\lengthtest { \paperwidth = 11in}}
    { \geometry{top=.5in,left=.5in,right=.5in,bottom=.5in} }
    {\ifthenelse{ \lengthtest{ \paperwidth = 297mm}}
        {\geometry{top=1cm,left=1cm,right=1cm,bottom=1cm} }
        {\geometry{top=1cm,left=1cm,right=1cm,bottom=1cm} }
    }

% Turn off header and footer
\pagestyle{empty}

% Redefine section commands to use less space
\makeatletter
\renewcommand{\section}{\@startsection{section}{1}{0mm}%
                                {-1ex plus -.5ex minus -.2ex}%
                                {0.5ex plus .2ex}%x
                                {\normalfont\large\bfseries}}
\renewcommand{\subsection}{\@startsection{subsection}{2}{0mm}%
                                {-1explus -.5ex minus -.2ex}%
                                {0.5ex plus .2ex}%
                                {\normalfont\normalsize\bfseries}}
\renewcommand{\subsubsection}{\@startsection{subsubsection}{3}{0mm}%
                                {-1ex plus -.5ex minus -.2ex}%
                                {1ex plus .2ex}%
                                {\normalfont\small\bfseries}}
\makeatother

% Don't print section numbers
\setcounter{secnumdepth}{0}

\setlength{\parindent}{0pt}
\setlength{\parskip}{0pt plus 0.5ex}

\newcommand\vv{\mathbf}
\newcommand\OO{\mathcal{O}}
\newcommand\JJ{\mathrm{J}}
\newcommand\dr{\partial}
\newcommand\grad{\mathrm{\mathbf{grad}}}
\renewcommand\div{\mathrm{div}}
\newcommand\rot{\mathrm{\mathbf{rot}}}


\begin{document}
\raggedright
\footnotesize
\begin{multicols}{3}


% multicol parameters
% These lengths are set only within the two main columns
%\setlength{\columnseprule}{0.25pt}
\setlength{\premulticols}{1pt}
\setlength{\postmulticols}{1pt}
\setlength{\multicolsep}{1pt}
\setlength{\columnsep}{2pt}

\begin{center}
     \Large{\underline{Acoustique en Fluide Réel}} \\
\end{center}

\section{Relations utiles}

$$\mathrm{Pr} = \frac{\nu}{\kappa}$$
$$\mu = \nu\rho$$
$$\kappa = \frac{\lambda}{\rho C_p}$$

\section{Thermodynamique}

Voir pp. 8-9-17-18-21-23.

\subsection{Transformation adiabatique}

$$PV^\gamma = cte$$
$$c^2 = \frac{\gamma}{\rho\chi_T}$$
$$C_p = \frac{P_0}{\rho_0T_0}\frac{\gamma}{\gamma-1}$$

\subsection{Gaz Parfait}

$$PV = nRT$$
$$\alpha = \beta = \nicefrac{1}{T} ~~,~~ \chi_T = \nicefrac{1}{P}$$

\subsection{Relations pour un fluide bivariant}

$$dS = \frac{C_p}{T}dT - \frac{P\beta}{\rho}\chi_TdP$$
$$dT = -\frac{T}{\rho\chi_TP}d\rho + \frac{T}{P}dP$$
$$d\rho = \rho\chi_T(dP - P\beta dT)$$

\section{Équations fondamentales}

Voir equations (2.1-2-3) ainsi que le chapitre 1.

\subsection{Conservation de la masse}

$$\frac{\dr\rho}{\dr t} + \mathrm{div}(\rho\vv{v}) = 0$$

\subsection{Conservation de l'énergie}

$$\rho T\frac{DS}{Dt} = \lambda\Delta T + \OO(||v||^2)$$

\subsection{Équation de Navier-Stokes}

$$\rho\frac{D\vv{v}}{Dt} = -\grad P + \mu\vv{\Delta}\vv{v} + (\eta + \frac{\mu}{3})\grad\div\vv{v}$$

$$\rho\frac{D\vv{v}}{Dt} = -\grad P + (\eta +\frac{4}{3}\mu) \grad\div\vv{v} - \mu\rot\rot\vv{v}$$

\subsection{Loi de Fourier}

$$\varphi = - \lambda\grad T$$

$\varphi$ : flux de chaleur

\subsection{Équations du Problème Acoustique}

$$\left\{\begin{array}{l}
		\frac{\gamma}{c_0}\frac{\dr p}{\dr t} - \frac{\rho_0c_0}{T_0}\frac{\dr\tau}{\dr t} + \rho_0c_0\div(\vv{v_l}) = 0\\
		\left[\frac{1}{c_0} \frac{\dr}{\dr t} - l_v\Delta\right]\vv{v_l} = -\frac{1}{\rho_0c_0}\grad p\\
		\left[\frac{1}{c_0}\frac{\dr}{\dr t} - l'_v\Delta\right]\vv{v_v} = \vv{0}\\
		\left[\frac{1}{c_0}\frac{\dr}{\dr t} - l_h\Delta\right]\tau = \frac{1}{\rho_0c_0C_p}\frac{\dr p}{\dr t}
\end{array}\right.$$

\subsection{Équations du Problème Acoustique}

Pour le détail des approximations formulées, voir diapos et les équations (2.44 à 49).  
Au premier ordre de $k_0l_{vh} << 1$ et pour $\xi = p,~\vv{v_l},~\mathrm{ou}~\tau$ : 

$$\left\{\begin{array}{l}
		(\Delta + k_v^2) \vv{v_v} = 0\\
		(\Delta + k_a^2)\xi_a \approx 0\\
		(\Delta + k_h^2)\xi_h \approx 0
\end{array}\right.
~~,~~
\begin{array}{l}
k_a = k_0\sqrt{1-jk_0l_{vh}}\\
k_v = \sqrt{\frac{-jk_0}{l'_v}}\\
k_h = \sqrt{\frac{-jk_0}{l_h}}
\end{array}
$$

\section{Équations d'ondes}

\subsection{Fluide visqueux non conducteur}

$$\left[1 + \frac{l_v}{c_0}\frac{\dr}{\dr t}\right]\Delta p - \frac{1}{c_0^2}\frac{\dr^2 p}{\dr t^2} = 0$$

$$k \approx k_0 - j\frac{l_v}{2}k_0^2$$

\subsection{Fluide visqueux et conducteur}

Voir pp. 24-25.

$$\underbrace{\left[\frac{1}{c_0}\frac{\dr}{\dr t} - l_h\Delta \right]}_{entropique}
\underbrace{\left[\frac{1}{c_0^2}\frac{\dr^2}{\dr t^2} - \left(1+l_{vh}\frac{1}{c_0}\frac{\dr}{\dr
t}\right)\Delta\right]}_{acoustique}
p \approx 0$$

\subsection{Avec relaxation thermique moléculaire}
Voir p. 29 équation (2.34-35).

$$\left[\Delta + k_0^2\left(1-jk_0l_{vh}-\sum_qD_q\frac{j\omega\theta_q}{1+j\omega\theta_q}\right)\right]p =0$$

$$D_q = \frac{(\gamma-1)C_{V_q}^v}{C_p}$$


\subsection{Atténuation monoatomique}

Voir pp. 26-27.
$$(\Delta + k^2)p = 0 ~~,~~ k = k_0 - j\frac{l_{vh}}{2}k_0^2$$

\subsection{Relations $\vv{v_l} \leftrightarrow \tau \leftrightarrow p$ monoatomique}

$$\left\{\begin{array}{l}
		p_a \approx \tau_a\rho_0 C_p\\
		p_h \approx 0
\end{array}\right. ~~,~~
\left\{\begin{array}{l}
	\frac{\dr\vv{v_{la}}}{\dr t} \approx - \frac{\gamma}{\gamma - 1} \frac{P_0}{\rho_0T_0}\grad\tau_a\\
	v_lh \approx \frac{\gamma P_0}{\rho_0c_0T_0}l_h\grad\tau_h
\end{array}\right.
$$

\subsection{Conditions à une interface}

Voir équations (2.40 à 43).

$$\vv{v_{\textbackslash\textbackslash}}(\vv{r_f},t) = \vv{v_{p_{\textbackslash\textbackslash}}}(\vv{r_f},t)$$
$$\tau(\vv{r_f},t) = 0$$

\section{Longueurs Caractéristiques}

\subsection{Longueurs $l_v$, $l'_v$, $l_h$ et $l_{vh}$}

Voir équations (2.8-9-10) et paragraphe (2.21.2).

$$\begin{array}{lcl}
	l_v = \frac{1}{\rho_0c_0}\left(\frac{4}{3}\mu + \eta\right) & ~,~ &  l'_v = \frac{\mu}{\rho_0c_0}\\
l_h = \frac{\lambda}{\rho_0c_0C_p} & ~,~ & l_{vh} = l_v = (\gamma - 1)l_h
\end{array}$$

\subsection{Couches limites thermiques et visqueuses}

Voir p.31, équations (2.38-39).

$$\delta_\nu = \sqrt{\frac{2\mu}{\omega\rho}} = \sqrt{\frac{2\nu}{\omega}}$$
$$\delta_\kappa = \sqrt{\frac{2\lambda}{\omega\rho C_p}} = \sqrt{\frac{2\kappa}{\omega}} = \frac{\delta_\nu}{\mathrm{Pr}}$$ 

\section{Propagation dans les guides}

\subsection{Équation de propagation}

Voir équations (2.77-78).

$$\dr^2_{xx} p + k^2p = 0$$
$$k = k_0\sqrt{1+\frac{f_\nu+(\gamma-1)f_\kappa}{1-f_\nu}}$$

\subsection{Fonctions visqueuse et thermique}

\subsubsection{Tuyau Cylindrique (rayon $R$)}

$$f_{\nu,\kappa} = \frac{2}{k_{v,h}R} \frac{\JJ_1(k_{v,h}R)}{\JJ_0(k_{v,h}R)}$$

\subsubsection{Fente (largeur $h$)}

$$f_{\nu,\kappa} = \frac{\mathrm{tanh}\left(\frac{(1+j)h}{2\delta_{\nu,\kappa}}\right)}{\left(\frac{(1+j)h}{2\delta_{\nu,\kappa}}\right)}$$

\subsubsection{Tuyau à section carrée}

Voir p. 41 équation (2.87).

% You can even have references
%\rule{0.3\linewidth}{0.25pt}
%\scriptsize
%\bibliographystyle{abstract}
%\bibliography{refFile}
\end{multicols}
\end{document}
